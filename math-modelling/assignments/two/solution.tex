% \documentclass[11pt]{article}
\documentclass[12pt,a4paper,oneside,draft]{article}
\usepackage[margin=1in]{geometry}
\usepackage{amsmath}
\usepackage{amsthm}
\usepackage{amssymb}
\usepackage{graphicx}
\usepackage{hyperref}
\usepackage[latin1]{inputenc}
\usepackage{mathtools}
\DeclarePairedDelimiter{\ceil}{\lceil}{\rceil}
\DeclarePairedDelimiter{\floor}{\lfloor}{\rfloor}

\title{MTH 361, Homework Assignment 2} 
\author{Nurseiit Abdimomyn -- 20172001}
\date{21/04/2020}

\begin{document}
\maketitle

\begin{enumerate}
  \item
    \begin{itemize}
      \item
        Find the adjacent matrix of network (a).
        \newline
        \newline
        $A = \begin{pmatrix}
              0&1&0&0&1\\
              0&0&1&0&0\\
              1&0&0&0&1\\
              0&1&1&0&0\\
              0&0&0&0&0\\
            \end{pmatrix}$
        \newline
      \item
        Find the incident matrix of network (b).
        \newline
        \newline
        $B = \begin{pmatrix}
              1&0&1&0&0\\
              0&1&1&0&0\\
              0&0&0&1&0\\
              0&1&1&1&1\\
            \end{pmatrix}$
        \newline
      \item
        Find the adjacent matrix for the network generated when we project
        the network (b) into its black vertices.
        \newline
        \newline
        $A' = \begin{pmatrix}
              0&0&1&0&0\\
              0&0&1&1&1\\
              1&1&0&1&1\\
              0&1&1&0&1\\
              0&1&1&1&0\\
            \end{pmatrix}$
        \newline
    \end{itemize}
  \item
    \begin{itemize}
      \item
        A 3-regular graph must have an even number of nodes.
        \begin{proof}
          By Handshaking lemma we have sum of all degrees: $$\sum_{v \in V} deg(v) = 2 * |E|$$
          and by the defition of regular graphs, 3-regular graph has sum of all degrees
          $$\sum_{v \in V} deg(v) = 3 * |V|.$$
          Thus, we have $$3 * |V| = 2 * |E|$$ which implies that $|V| = 2 * k$ for some $k$.
        \end{proof}
      \item
        The average degree of a tree is strictly less than 2.
        \begin{proof}
          Via Handshaking lemma, the average degree of a graph is as follows:
            $$c = \frac{1}{|V|} * \sum_{v \in V} deg(v) = \frac{2 * |E|}{|V|}.$$
          By definition, the following holds true for all trees: $$|E| = |V| - 1.$$
          Thus, by substituting $|V|$:
            $$c = \frac{2 * (|V| - 1)}{|V|} = 2 - \frac{1}{|V|} < 2$$
        \end{proof}
    \end{itemize}
  \item
    Consider a network which is simple (it contains no multiedges or self-edges)
    and consists of $n$ nodes in a single component.
    \newline
    (i) What is the maximum possible number of edges it could have?
    \newline
    (ii) What is the minimum possible edges if could have?
    \newline
    Explain how you give the answer by providing the corresponding figures of networks.
    \begin{itemize}
      \item [(i) ]
        One could draw an edge between every of the $n$ nodes of the graph to
        form a \textit{complete graph} with $|E| = \frac{n * (n - 1)}{2}$.
        \newline ToDo draw graph
      \item [(ii) ]
        One could form a \textit{tree graph} with $n$ nodes to get $|E| = n - 1$.
        \newline ToDo draw tree
    \end{itemize}
  \item
    \begin{itemize}
      \item [(i) ] How do $n$, $m$, and $f$ change when we add a single vertex to such a network 
      along with a single edge attaching it to an existing vertex?
      \newline
        $$n \implies n + 1; m \implies m + 1; f \implies f;$$
      One can't form any "face" with $1$ new edge and $1$ new node only.
      \newline
      \newline
      \item [(ii) ] How do $n$, $m$, and $f$ change when we add a single edge between two existing vertices
      (or a self-edge attached to just one vertex), in such a way as to maintain planarity of the network?
      \newline
        $$n \implies n; m \implies m + 1; f \implies f + 1;$$
      By adding an edge while maintaining planarity of the graph we will bound a new area and form a new "face".
      \newline
      \newline
      \item [(iii) ] What are the values of $n$, $m$, and $f$ for a network with a single vertex and no edges?
        $$n \implies 1; m \implies 0; f \implies 1;$$
      With no "faces" except the outer one.
      \newline
      \newline
      \item [(iv) ] Hence by induction prove a general relation between n, m, and f for all
      connected planar networks.
      \newline
      \newline
      Let's prove Euler's identity for planar graphs as $n - m + f = 2$, where
      $n = |V|, m = |E|, f = |\textit{faces}|$.
      \newline
      (1.) Basic step of induction is given in (iii):
        $$n \implies 1; m \implies 0; f \implies 1;$$
        $$\text{so, } n - m + f = 1 - 0 + 1 = 2; \square$$
      \newline
      (2.) Induction step is given in (i) and (ii) by assuming $n - m + f = 2$ is \textit{true}:
        $$\text{(i): } n \implies n + 1; m \implies m + 1; f \implies f;$$
        $$\text{so, } (n + 1) - (m + 1) + f = n - m + f = 2;$$
        $$\text{(ii): } n \implies n; m \implies m + 1; f \implies f + 1;$$
        $$\text{so, } n - (m + 1) + (f + 1) = n - m + f = 2; \square$$
      \newline
      \item [(v) ] Now suppose that our network is simple. Show that the mean degree $c$ of a simple,
      connected, planar network is strictly less than \textit{six}.
      \newline
      \begin{proof}
        By Handshaking lemma, we know that mean degree is
          $$c = \frac{1}{|V|} * \sum_{v \in V} deg(v) = \frac{2 * |E|}{|V|} = \frac{2 * m}{n},$$
        and we proved $n - m + f = 2$ in (iv).
        \newline
        Similar to Handshaking lemma, we know for sum of degress of all faces:
          $$\sum_{i} deg(f_i) = 2 * |E| = 2 * m.$$
        From there, because our graphs are all simple, the smallest possible degree of a face
        would be $3$, so:
          $$\sum_{i} 3 \leq \sum_{i} deg(f_i) \implies 3 * f \leq 2 * m.$$
        Thus, by solving for $f$ in $n - m + f = 2 \implies f = 2 + m - n$ we get:
          $$3 * f \leq 2 * m \implies 3 * (2 + m - n) \leq 2 * m \implies m \leq 3*n - 6.$$
        Further, by substituting the above to the equation for mean degree $c$:
          $$c = \frac{2 * m}{n} \leq \frac{2*(3*n - 6)}{n} \implies c \leq 6 - \frac{12}{n}.$$
        Which for all $n \neq 0$ it's true that $c < 6.$
      \end{proof}
    \end{itemize}
\end{enumerate}

\end{document}
