\documentclass[12pt]{article}
\usepackage{amsmath}
\usepackage{amssymb}
\usepackage{graphicx}
\usepackage{hyperref}
\usepackage[latin1]{inputenc}
\usepackage{mathtools}
\DeclarePairedDelimiter{\ceil}{\lceil}{\rceil}
\DeclarePairedDelimiter{\floor}{\lfloor}{\rfloor}

\title{CSE232 Assignment 4} 
\author{Nurseiit Abdimomyn -- 20172001}
\date{27/12/2018}

\begin{document}
\maketitle

\begin{enumerate}
  \item
    $a = 6, b = 2, c = 3$. So, $6 | (2 * 3), 6 \nmid 2, 6 \nmid 3$.
  \item
    Because $7$ and $123$ are coprime, we have by Euler's Theorem:
    \newline
    $\phi (7) = 6$ and $123^{456} \equiv 123^{456 \mod \phi(7)}$ $(\mod 7)$.
    \newline
    So, $123^{456} \equiv 123^{456 \mod 6} \equiv 123^0 \equiv 1$ $(\mod 7)$.
    
  \item
    It's sufficient to check for $6$ congruent clasess. \newline
    For $n \equiv 0, 0^2 \mod 6 \equiv 0 \neq 2$. $(\mod 6)$\newline
    For $n \equiv 1, 1^2 \mod 6 \equiv 1 \neq 2$. $(\mod 6)$\newline
    For $n \equiv 2, 2^2 \mod 6 \equiv 4 \neq 2$. $(\mod 6)$\newline
    For $n \equiv 3, 3^2 \mod 6 \equiv 3 \neq 2$. $(\mod 6)$\newline
    For $n \equiv 4, 4^2 \mod 6 \equiv 4 \neq 2$. $(\mod 6)$\newline
    For $n \equiv 5, 5^2 \mod 6 \equiv 1 \neq 2$. $(\mod 6)$\newline
  \item
    The strongly connected components are: \newline
    $\{i\}$ \newline
    $\{a, b, c\}$ \newline
    $\{d, e, g, h\}$ \newline
    $\{f\}$ \newline
  \item
    \begin{itemize}
      \item [(a) ]
        Has an Euler path because it has exactly two vertices with odd degree. $\{d, f\}$
      \item [(b) ]
        Doesn't have Euler circuit -- not all vertices have even degree.
      \item [(c) ]
        It has a Hamilton Path as: $\{a, b, e, f, g, c, d\}$
      \item [(d) ]
        By Dirac's Theorem we know that in a graph with $3 \leq n$ vertices, if each vertex has $n/2 \leq deg(v)$, \newline
        then the graph has a Hamilton circuit. \newline
        However, this theorem is not neccessary but it is sufficient. \newline
        A quick manual check gives us that this graph doesn't have a Hamilton circuit.      
    \end{itemize}
    \item
      We know that every graph that doesn't have a cycle of odd length is bipartite. \newline
      All trees are acyclic so we might say that they have a cycle of length $0$ which is even. \newline
      Thus, all trees are bipartite.
\end{enumerate}

\end{document}
