\documentclass[12pt]{article}
\usepackage{amsmath}
\usepackage{amssymb}
\usepackage{graphicx}
\usepackage{hyperref}
\usepackage[latin1]{inputenc}
\usepackage{mathtools}
\DeclarePairedDelimiter{\ceil}{\lceil}{\rceil}
\DeclarePairedDelimiter{\floor}{\lfloor}{\rfloor}

\title{CSE232 Assignment 2} 
\author{Nurseiit Abdimomyn -- 20172001}
\date{09/10/2018}

\begin{document}
\maketitle

\begin{enumerate}
  \item
    $(p \to q) \to (\neg q \to \neg p) \equiv \neg (p \to q) \lor (\neg q \to \neg p) \equiv$
    \newline
    $\neg (\neg p \lor q) \lor (q \lor \neg p) \equiv (p \land \neg q) \lor (q \lor \neg p) \equiv$
    \newline
    $((p \land \neg q) \lor q) \lor ((p \land \neg q) \lor \neg p) \equiv$
    \newline
    $((p \lor q) \land (q \lor \neg q)) \lor ((p \lor \neg p) \land (\neg q \lor \neg p)) \equiv$
    \newline
    $(p \lor \neg p) \lor (q \lor \neg q) \equiv T$
  \item
    $\exists x \forall y (xy < 0 \to x < y)$ where the domain for $x, y$ is $\mathbb{R}$
    \newline
    Let's find a counterexample:
    \newline
    Let $y = -1$ and by definition $xy < 0$. So, $-x < 0$ which is the same as $x > 0$. Thus $y < x$, which is a contradiction. Q.E.D.
  \item
    Let $n$ be an integer.  Prove or disprove the following:  If $3n$ is odd, then $n$ is odd.
    \newline
    $f(3n) \to f(n)$ where $f(x)$ is $T$ for $x$ is odd. $F$ otherwise.
    \newline \newline
    Proof by a contrapositive:
    \newline
    $\neg f(n) \to \neg f(3n)$
    \newline
    Let's write $n = 2k$, then it's clear that $3n = 6k$. Which is also even as $6k = 2*(3k)$ Q.E.D.
  \item
    $f(x) = f(y) \to x = y;$ \newline
    $g(x) = g(y) \to x = y;$ \newline
    So is true for $f(g(x)) = f(g(y)) \to f(x) = f(y) \to x = y;$ \newline
    Q.E.D
  \item
    $\forall n \in \mathbb{Z} (\ceil{n/2} + \floor{n/2} = n)$ \newline
    Let's consider the cases when $n$ is even and $n$ is odd. \newline \newline
    For $n$ is $even$: It is obvious even without proving that the equation above holds in this particular case. \newline
    For $n$ is $odd$: let's take $n = 2k + 1$, then $n/2 = k + 1/2$; \newline
    So, $\floor{k + 1/2} = k$ and $\ceil{k + 1/2} = k + 1$. \newline
    Thus, $\ceil{k + 1/2} + \floor{k + 1/2} = k + k + 1 = 2*k + 1$ which by our definition is equal to $n$. \newline Q.E.D.
  \item
    When finding how many integers from $1$ to $n$ are divisible by some integer $k$, one can prove that $\floor{n/k}$ is the answer. \newline \newline
    So by applying Inclusion-Exclusion Principle here we can see that the number of integers from $1$ to $1000$ which are divisible by either $4$ or $9$ is nothing but: \newline
    $\floor{1000/4} + \floor{1000/9} - \floor{1000/(4*9)} = 250 + 111 - 27 = 334$
  \item
    Via Inclusion-Exclusion Principle the answer will be as follows: \newline
    $\floor{1000/4} + \floor{1000/6} - \floor{1000/x}$ where $x = LCM(4, 6) = 12;$ \newline
    So, $250 + 166 - 83 = 333$.
\end{enumerate}

\end{document}
